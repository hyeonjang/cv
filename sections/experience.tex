% YAAC Another Awesome CV LaTeX Template
%
% This template has been downloaded from:
% https://github.com/darwiin/yaac-another-awesome-cv
%
% Author:
% Christophe Roger
%
% Template license:
% CC BY-SA 4.0 (https://creativecommons.org/licenses/by-sa/4.0/)
%Section: Work Experience at the top
\sectionTitle{경험}{\faSuitcase}
%\renewcommand{\labelitemi}{$\bullet$}
\begin{experiences}
  \experience
    {2021.03 $\sim$ 현재}{연구원}{컴퓨터그래픽스 연구실}{광주과학기술원}    
    {2022.01} {석사논문연구주제 (Adaptive weighted Laplacian regulazation)
                \begin{itemize}
                  \setlength\itemsep{0.5em}
                  \item Differentiable Rendering을 통한 mesh 최적화 연구
                  \newline *Differentiable Rendering은 Inverse Rendering의 한 종류로 rendering에서 최적화기법을 사용해 입력값를 구하는 기법
                  \newline **Differentiable Rendering은 목표 이미지와 렌더링 이미지 사이의 loss 함수를 설정하고, 해당 함수에서 입력값에 대한 gradient descent 방법으로 최적화된 입력값을 구해냄
                  \item 해당 주제로 출간된 EPFL RGL의 SIGGRAPH Asia 2021 논문은 Stochastic Gradient Descent를 활용한 최적화 단계에서 discrete Laplacian Regularization을 생성, 적용하여 SGD의 variance를 줄이는 방법을 고안해냄
                  \item 개인연구에서는 Laplacian Regularzation의 성능을 개선하는 적응형 방법을 연구개발하여, 현재 국제학술지 투고 준비중
                \end{itemize}
                }{C++17, CMake, gcc, Mitsuba2, Geometry-Central, Nvdiffrast, Python, Pytorch, Pybind11, vscode, Ubuntu20.04}
  \emptySeparator
  \experience
    {2021.03 $\sim$ 현재}{연구원}{컴퓨터그래픽스 연구실}{광주과학기술원}
    {2021.03} {미술관 XR트윈 프로젝트, 문화기술체육관광부 지정과제
                \begin{itemize}
                  \setlength\itemsep{0.5em}
                  \item 광주광역시 광주시립미술관 실제공간과 동일하게 가상화하여 미술관 객체에 대한 XR트윈을 생성하는 프로젝트
                  \item 해당 프로젝트에서 미술관에 대한 스캔데이터를 취득하여 전시관에 대한 mesh와 texture를 생성하는 파이프라인을 구축
                  \item 외부 라이브러리를 활용하여, 스캔데이터(RGBD 이미지, 카메라 포즈)를 후처리하는 코드 구현
                  \item 외부 라이브러리를 활용하여, 후처리된 스캔데이터를 mesh를 생성 후, hole filling, remeshing 등의 Geometry processing 알고리즘을 적용하여 최적화하는 코드 와 mesh의 vertex color를 texture로 변환하는 코드 구현
                  \item 해당 프로그램을 평가받기 위한 소프트웨어 공인인증시험 준비 및 통과
                \end{itemize}
                }
                {Open3D, Geogram, tbb, C++20, CMake, gcc, vscode, Ubuntu20.04}
  \emptySeparator
  \experience
    {2020.01 $\sim$ 07}{학부연구생}{컴퓨터그래픽스 연구실}{성균관대학교}
    {2020.06}{GPU 기반 실시간 멀티바운스 주변폐색 렌더링, KCGS2020
              \vspace{0.2em}
              \begin{itemize}
                \setlength\itemsep{0.5em}
                \item 연구실 내 자체 실시간 렌더링 프레임워크에 GPU 기반의 광선추적법 기능 구현
                \item Mesh intersection 체크를 위해, 광선추적법에서 GPU에서 활용가능한 인덱스 기반의 kDtree 코드 작성
                \item kDtree 탐색의 성능 개선을 위한 kDtree push down 알고리즘 작성
                \item Fragment Shader에 GPU 광선추적법 구현, 해당 기능을 발전 ambient occlusion 코드 구현 및 학부생 논문초록 작성
              \end{itemize}
              }{C++14, VisualStudio, OpenGL4.6, Windows10}
  \emptySeparator
\end{experiences}
